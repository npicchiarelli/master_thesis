\documentclass[a4paper, notitlepage]{report} % type of document
\usepackage[T1]{fontenc}      % font encoding
\usepackage[utf8]{inputenc}   % sepcial chars from keyboard
\usepackage[english]{babel}   % document language
\usepackage{amsmath}          % math symbols
\usepackage{physics}

\usepackage[backend = biber, style=numeric]{biblatex}         % bibliograpy


\addbibresource{summary.bib}
\title{Thesis Title}
\author{Candidate: Niccolò Picchiarelli \\
Supervisor: Prof. Stefano Palagi\\
Internal Supervisor: Prof. Riccardo Mannella}

\begin{document}
	\vspace{-4cm}
	\maketitle
	\emph{Active Matter} is a term that refers to systems, both at macroscopic and microscopic scales, which can be described as sets of individual constituents, often called active or self-propelled particles, that have the ability of taking energy from the environment or an internal source to convert it to work. 
	Such systems show peculiar behaviors, both individual and collective, due to their intrinsic far-from-equilibrium physical properties, as well as interactions that may occur between active particles \cite{menon_active_2010, ramaswamy_active_2017}.
	
	Among various active matter models, Active Brownian Particles (ABPs) stood out for its simplicity and capability of showing interesting collective dynamics.
	This model is obtained coupling a classic Brownian motion with a deterministic self-propulsion velocity along a preferred direction, combined in the equations (2D)
	\begin{equation}
		\dot{x} = v \cos{\theta} + \sqrt{2D_t}\dd{W_x} , \quad \dot{y} = v \sin{\theta} + \sqrt{2D_t}\dd{W_y}, \quad \dot{\theta} = \sqrt{2D_r}\dd{W_{\theta}}
	\end{equation}
	where $\dd{W}$ is the derivative of a zero mean, unit variance Wiener process.
	
	It is possible to show that a model like this can mimic the features of a real system of active particles \citeauthor{bechinger_active_2016}, leading the way to the discoveries of new properties of active matter, that can be used both to understand the ongoing physics behind natural systems and to create new technologies based upon such features.
	The aim of CELLOIDS, the project inside of which this work has taken place, is to build a micro-scale intelligent robotic system utilizing self-diffusio-phoretic Janus spheres enclosed inside phospholipids GUVs (Giant Unilamellar Vesicles).
	Janus spheres are micromotors with an inert hemispheres and a catalytic one, that, catalyzing the decomposition of some fuel, create a gradient in the product concentration field, hence causing a phoretic flow around them that makes them move with a preferred propulsion direction.
	Under certain assumptions such micromotors behave as Active Brownian particles, showing fascinating collective motions.
	
	The main objective of this thesis work is to build a model, a simulation framework and a set of analysis tools to better understand how interactions and several other model features can change collective behaviors of an ABPs ensemble.
	
	To do this, we implemented interactions inside an existing simulation code, keeping an eye on the speed of the calculations.
	Interactions in active particles systems are classified in two categories: non-aligning, which act only on particles' positions, and aligning, which change particles' orientations and depend on them too.
	It is known from experimental literature \cite{singh_pair_2024}, that a pair of Pt-silica Janus spheres undergo some interaction torques when close to each other, so a model aiming to mimic real world behaviors need to involve some aligning interaction.
	
	Our way to implement aligning interaction in an interacting active particles model is the main novelty of this work: instead of treating positions and orientations separately, we used classic central potentials like $r^{-2}$ and Lennard Jones, but applying them to an off center position on the disk representing the particle.
	The shifted position is obtained starting from the center and translating of an amount $\alpha R$, where $R$ is particle's radius and $\alpha \in [-1,1]$, in the direction of self-propulsion.
	This is in a qualitative agreement with the fact that the interaction mechanism is the only that for sure makes particles interact and that phenomenon has an evident asimmetry in the self-propulsion direction.
	This model not only changes the positions from which forces are computed but also inserts a lever arm in the interaction, causing an interaction torque to appear among couples of particles.
	
	Some authors, like \citeauthor{martin-gomez_collective_2018}, showed that, inserting an aligning torque in an ABPs system can make it transition from a disordered state to an ordered one where all particles swim in the same direction, similarly to what happens in a Vicsek model \cite{vicsek_novel_1995}.
	The model in \cite{martin-gomez_collective_2018} involves a non-superposition force and an explicitly aligning torque.
	Here, we show that the transition to a flocking state has a similar behavior also in the case of our model when a purely repulsive potential is applied.
	
	In order to study how the simulation features change the global properties of the ensemble we developed a suite of analysis tools.
	To better understand flocking transitions in the system we used the global polarization, defined as 
	\begin{equation}
		P = \frac{1}{N} \abs{\sum_{k=1}^{N} e^{i \theta_k(t)}} 
	\end{equation}
	where $\theta_k$ is the orientation of $k$-th particle with respect to the $x$-axis.
	We know that, with potentials that allow contact between particles, such as Lennard Jones, systems like the one studied here tend to cluster.
	We then used a clustering algorithm to study how the ensemble separates into sparse particles and living crystalline lattices.
	The chosen algorithm is DBSCAN, which, with the right parameters, can mimic a common-sense definition of \emph{cluster}.
	We used maximum cluster size and cluster number to study how order in particles' positions establishes.
	Once the system is divided in clusters, one can study how the single clusters are aligned, restricting the polarization computation to the single groups.
	The last tool, that we used to fathom the amount of long-range order in the ensemble, is the well known pair distribution function, radial distribution function, or simply $g(r)$, which is defined \cite{hansen90a} as 
	\begin{equation}
		\left\langle \frac{1}{N} \sum_{i=1}^{N} \sum_{j=1}^{N}{}' \delta (\mathbf{r} - \mathbf{r}_j + \mathbf{r}_i) \right\rangle = \rho g(r)
	\end{equation}.
	
	With all of this, we have been able to better understand the effect of velocity, velocity distribution, orientational coupling $\alpha$ and angular velocity when a LJ potential is applied.
	
	As a final task, we tried to develop a machine learning tool, based on Graph Neural Networks, that, starting from particles positions and orientations, tries to predict their velocity, separating self-propulsion and interaction, to infer the interaction potential, in the same spirit of \cite{ruiz-garcia_discovering_2024}.
	\printbibliography
\end{document}